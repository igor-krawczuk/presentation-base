%!TeX program = xelatex
\documentclass[10pt]{beamer}

\usetheme{metropolis}
\usepackage{appendixnumberbeamer}
\usepackage[english]{babel}
%\usepackage[utf8]{inputenc}

\usepackage{booktabs}
\usepackage[scale=2]{ccicons}

\usepackage{pgfplots}
\usepgfplotslibrary{dateplot}

\usepackage{xspace}
%\newcommand{\themename}{\textbf{\textsc{metropolis}}\xspace}


\usepackage[backend=biber, style=authoryear,sorting=none,backref,backrefstyle=none]{biblatex} % use biblatex
\addbibresource{presi.bib}
%\usepackage{dejavu} % load my favourite KISS font
\usepackage{ETbb} % my favourite fancy font
\renewcommand*\familydefault{\sfdefault} %% Only if the base font of the document is to be sans serif
\usepackage[T1]{fontenc}
%\usefonttheme{default} %TODO: what do these actually do?
\usefonttheme{professionalfonts}
%\usefonttheme{serif}
\usepackage{mathpazo,euler}
\usepackage{fontspec}
%math stuff
\usepackage{amsmath}
\usepackage{amssymb}
\usepackage{bm}
% graphics
\usepackage{graphicx}

\title{A title}
%\subtitle{A subtitle}
\date{\today}
\author{Igor Krawczuk}
%\institute{Center for modern beamer themes}
% \titlegraphic{\hfill\includegraphics[height=1.5cm]{logo.pdf}}

\begin{document}

\maketitle



% for fragile explanation, see https://tex.stackexchange.com/questions/317633/macro-allowframebreaks-as-default-returns-error-when-loaded-with-fragile/317712#317712
\begin{frame}[fragile]{Introducing myself}
\end{frame}
%{%
%\setbeamertemplate{frame footer}{My custom footer}
%\begin{frame}[fragile]{Frame footer}
%    \themename defines a custom beamer template to add a text to the footer. It can be set via
%    \begin{verbatim}\setbeamertemplate{frame footer}{My custom footer}\end{verbatim}
%\end{frame}
%}

\begin{frame}{Structure of this talk}
  \setbeamertemplate{section in toc}[sections numbered]
  \tableofcontents[hideallsubsections]
\end{frame}

\section{Introduction}
\begin{frame}[fragile]{Motivation}
  \cite{Knuth92}
  \cite{greenwade93}
\end{frame}
\begin{frame}[fragile]{Why do we care?}
\end{frame}
\begin{frame}[fragile]{What exists already?}
\end{frame}

\section{Our thing}

\begin{frame}{Background to our thing}
\end{frame}


\begin{frame}{Our actual thing}
\end{frame}

\begin{frame}{Evaluation of our thing}
\end{frame}
\begin{frame}{Discussion of our thing}
\end{frame}
\section{Conclusion}
\begin{frame}{Recapping our thing}
\end{frame}
\begin{frame}{Questions answered by our thing}
\end{frame}
\begin{frame}{Questions opened/left open by our thing}
\end{frame}

\begin{frame}{Summary}

  Get the source of this theme and the demo presentation from

  \begin{center}\url{github.com/matze/mtheme}\end{center}

  The theme \emph{itself} is licensed under a
  \href{http://creativecommons.org/licenses/by-sa/4.0/}{Creative Commons
  Attribution-ShareAlike 4.0 International License}.

  \begin{center}\ccbysa\end{center}

\end{frame}

\begin{frame}[standout]
  Questions?
\end{frame}

\appendix


\begin{frame}[fragile]{Backup slides}
  \nocite{*}
%  Sometimes, it is useful to add slides at the end of your presentation to
%  refer to during audience questions.
%
%  The best way to do this is to include the \verb|appendixnumberbeamer|
%  package in your preamble and call \verb|\appendix| before your backup slides.
%
%  \themename will automatically turn off slide numbering and progress bars for
%  slides in the appendix.
\end{frame}

\begin{frame}[allowframebreaks]{References}
  \printbibliography

\end{frame}

\end{document}
